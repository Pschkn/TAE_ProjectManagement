Risk analysis is a process which enables the analysis of risks, associated
within  a project. A Risk can be generally defined as the probability of
something going wrong, and the negative consequences if it does. However, it is
hard to find all the risks, which can occur, in a project. At first it should be
recognised that a risks exists as a consequence of uncertainty. For this reason
the risk analysis process will help to identify potential problems that may
occur. Such a risk analysis can be useful in several situations: 
\begin{itemize}
\item To help to anticipate and neutralise possible problems when planning
projects.
\item To decide wether to continue with the project or not.
\item To improve safety and manage potential risks in the workplace.
\end{itemize}


\section{Identifying the Risks}
As one of the first steps in Risk Analysis it is to identify the existing and
possible problems occur. Some of these areas and threats, which might have an
impact on this project, are listet below:
\begin{itemize}
\item Project Members - Illness, injury, or another reason leading to a loss of
a project member.
\item Operational � Delays in deliveries.
\item Reputational � Loss of customer or employee confidence.
\item Project � Taking too long on concluding key tasks, or experiencing
issues with product or service quality, Goal not achieved.
\item Financial � Budget exhausted, Business failure or non-availability of
funding.
\end{itemize}

\section{Estimate Risks}
After some of the possible threats has been faced, the risk can be calculated
with both the likelihood of these threats being realised, and their possible
impact. One way of doing this is to make a estimation of the probability that
this threat occurrs multiplied by the amount it will cost. This leads to the
following equation which quantifies the risk:
\begin{equation}
Risk = Probability of Occurance \cdot Cost
\end{equation}
Additionally there are two possible kinds of processes:
\begin{itemize}
\item The total value of a risk of a series of processes that are executed
successively can be calculated as follows:
\begin{equation}
R_{Total} = R_{n} \cdot R_{n+1}
\end{equation}
\item The total value of a risk of parallel processes that are executed
concurrent can be calculated as follows:
\begin{equation}
R_{Total} = 1 - (1-R_{n}) \cdot (1 - R_{n+1})
\end{equation}
\end{itemize}

\textbf{\textcolor{red} {impact chart machen bzw zuerst unsre threats schaetzen
und dann einfach mal das risiko berechnen}}


Low impact/low probability � Risks in the bottom left corner are low level, and you can often ignore them.
Low impact/high probability � Risks in the top left corner are of moderate importance � if these things happen, you can cope with them and move on. However, you should try to reduce the likelihood that they'll occur.
High impact/low probability � Risks in the bottom right corner are of high importance if they do occur, but they're very unlikely to happen. For these, however, you should do what you can to reduce the impact they'll have if they do occur, and you should have contingency plans Add to My Personal Learning Plan in place just in case they do.
High impact/high probability � Risks towards the top right corner are of critical importance. These are your top priorities, and are risks that you must pay close attention to.


%https://www.mindtools.com/pages/article/newTMC_07.htm

%https://www.mindtools.com/pages/article/newPPM_78.htm

%http://www.fep.up.pt/disciplinas/PGI914/Ref_topico3/ProjectRAM_APM.pdf
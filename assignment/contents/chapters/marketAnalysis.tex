An important step in the development of a new product is to analyse the market.
This analysis not only includes the identification of competitors and their offered
technologies, but also the investigation of the demand on the product to develop
and its future progression. Another part of the market analysis is to determine
which hardware might be necessary for the product to develop. After the
identification of the needed parts, the offered products on the market are
compared regarding their capabilities and their price.

\section{Analysis of Competitors}
The analysis of competitors provides an overview of the available products on
the market, of their capabilities and of how widely these products are already
used. The following analysis is done by querying the internet and putting
altogether the relevant information. It not only focuses on well-known
manufacturers but also on scientific projects that might be in competition with
our company.

\subsection{Current Systems implemented in today's Cars}
There are only a few systems available that help the driver in leaving a parking
space. These systems exhibit a huge variety of autonomy. The manufacturers
Volvo, Audi and Lincoln sheet park assistance systems that take control over the
steering wheel when leaving a parking lot (see \cite{VolvoCarsSupport},
\cite{Lincoln} and \cite{AudiEspana}). While the steering is done autonomously, 
the driver has to operate the pedals on its own.
This kind of systems is mostly restricted to parallel parking.

Mercedes-Benz offers a more autonomous, but also more restricted way of assisted
parking. The Mercedes-Benz ``Parking Pilot'' is able to park and leave a parking
site autonomously. The quitting of the parking site is restricted to those
scenarios in which the Parking Pilot was also used to park the car (see
\cite{DaimlerParkingPilot}).

Tesla offers the ``Summon'' functionality implemented in its Model S and Model X. It
allows a driver to leave its car and park as well as retrieve it autonomously.
This feature is restricted to perpendicular parking only (see \cite{TeslaSummon}).

\subsection{Current Systems available from Suppliers}

The development of systems assisting a driver in parking and leaving a parking
lot can be illustrated by the evolution of the products originating from Robert
Bosch GmbH. While the early systems act as it was described for the
manufacturers Volvo, Audi and Bosch (see \cite{Bosch2013}),
 the current systems are now able to drive a car into and back out of a parking
site autonomously (see \cite{BoschFully}). Another future application of park
assistants is the Bosch ``Home Zone Park Assist''. It enables a driver to train its car for certain parking situations
(see 
\cite{BoschHomeZone}).
The car records a route that is driven and it is able to reproduce it even if
the starting point of the route to drive and the one of the recorded route are
slightly different.
On its trained way, the car is able to detect impediments and to react to them.

\subsection{Scientific Projects}
The earliest work on the current topic that could be found is the one of
\cite{Braun}. This team presents some analytical work as well on the system
design as on the necessary hardware to implement the desired functionality. In
addition, they also provide a detailed analysis on the structure of the needed
algorithms.

While it couldn't be determined if this project was brought to termination,
Roland Doloczki and Don Kevin Gaubitz produced a working prototype of an RC-Car
that autonomously leaves a parking space (see
\cite{Doloczki}).
To achieve their goal, Doloczki and Gaubitz used ultrasonic and infrared sensors
to sense the environment around parked vehicle.

\section{Future Development of the market}
It is obvious that the demand on systems that perform certain manoeuvres
autonomously will increase with the success of autonomous cars. But also in the
meantime till these cars make the breakthrough, there might be an increased need
for \acf{ADAS} like parking assistants. Following
McKinsey Inc., there will be three eras in the revolution of self-driving cars
(see \cite{McKinsey}).
The first era, starting from now and lasting till the late 2020s, is
characterised by the first autonomous vehicles being produced and by their
impact on the established car manufacturers. McKinsey states that the premium
makers will take an incremental approach to autonomous vehicles by implementing more
sophisticated ADAS. This assumption is supported by Statista, assuming that the
shipment of ADAS units will increase by more than 500\% in the time from 2012 to
2020 (see \cite{StatisticaADAS}).

One of the buzz words regarding future driver assistance systems is ``Valet
Parking'' which means that a car parks itself after the driver has left it and
that the car can be retrieved from its parking position without active control
of the driver. Therefore, ``Valet Parking'' needs the possibility of a car
autonomously leaving its parking site. A research project targeting on this
technology was announced by Daimler, Bosch and Car2go in the year 2015 (see
\cite {DaimlerValet}).

\section{Market Analysis regarding the needed Hardware}
The product to develop is based on the recognition of obstacles in the vehicle's
surroundings. The most common used sensors to gain an overview of a car's
ambiance are ultrasonic and laser sensors as well as cameras. Some representatives of
these sensors are introduced and compared in this chapter.

\subsection{Comparison of Ultrasonic Distance Sensors }
There exist a lot of ultrasonic distance sensors on the market that are intended
to be used in automotive applications. The chosen representatives of these all
exhibit a detection range of $1m$ or above. Their switching frequency, operation
temperature and price are compared in table \ref{tab:ultrasonic}.

\begin{table}
\centering
\captionsetup{justification=centering}
\footnotesize
\renewcommand{\arraystretch}{1.5}
\begin{tabular}{p{0.3\textwidth}|p{0.35\textwidth}|p{0.1\textwidth}|p{0.1\textwidth}}
Product & Key Features & Price & Retailer \\
\hline
SICK UM18-218161101 
& Range: 0.12 - 1.0m \newline 
	Operation temp.: -25\degree C - 70\degree C \newline 
	Switching freq.: 10Hz 
& 152.32\$ 
& \cite{tme} \\
PING))) Ultrasonic Distance Sensor 
& Range: 0.02 - 3m \newline 
	Operation temp.: 0\degree C - 70\degree C  \newline 
	Switching freq.: 10Hz 
& 24.99\$ 
& \cite{parallax}\\
LV-MaxSonar-EZ1 
& Range: 0.0 - 6.45m\newline 
	Operation temp.: -  \newline 
	Switching freq.: 20Hz  
& 23.36\$ 
& \cite{sparkfun} \\
AU003 
& Range: 0.08 - 1.2m\newline 
	Operation temp.: -20 \degree C - 70 \degree C  \newline 
	Switching freq.: 5Hz  
& 112.94\$ 
& \cite{autosen} \\
\end{tabular}
\caption{Compared Ultrasonic Distance Sensors}
\label{tab:ultrasonic}
\end{table}

While the low-cost sensors are appropriate for a proof of concept, they are not
suitable for an application under real conditions because either their operation
temperature lies only above freezing or it is not indicated in the datasheets.
The prices of the high-cost sensors are based on the ordering of small amounts
and might be renegotiated if higher volumes are commissioned.

\subsection{Comparison of Laser Distance Sensors}
Laser distance sensors are especially popular in the context of the obstacle
detection that is implemented by Google's self-driving car (see \cite{whitwam}).
Different representatives of this kind of sensor are contrasted in table
\ref{tab:laser}.
\begin{table}
\centering
\captionsetup{justification=centering}
\footnotesize
\renewcommand{\arraystretch}{1.5}
\begin{tabular}{p{0.3\textwidth}|p{0.35\textwidth}|p{0.1\textwidth}|p{0.1\textwidth}}
Product & Key Features & Price & Retailer \\
\hline
OID200 - OIDLCPKG/US
& Range: 0.03 - 2.0m \newline 
	Operation temp.: -25\degree C - 60\degree C \newline 
	Switching freq.: 11Hz 
& 115.69\$ 
& \cite{automation24} \\
Lidar Lite v3
& Range: up to 40m \newline 
	Operation temp.: -20\degree C - 60\degree C  \newline 
	Switching freq.: 10Hz 
& 159.15\$ 
& \cite{sparkfun2}\\
AL003 
& Range: 0.03 � 2m\newline 
	Operation temp.: -25\degree C - 60\degree C \newline 
	Switching freq.: 11Hz  
& 110.05\$ 
& \cite{autosen2} \\
\end{tabular}
\caption{Compared Laser Distance Sensors}
\label{tab:laser}
\end{table}

All of the presented sensors are designed for the use in automotive applications
and therefore fulfil the requirements for our project. In contrast to the
ultrasonic sensors, there are no low-cost laser distance sensors that are
suitable for the use in a proof of concept.

\subsection{Review on available Camera Sensors}
Comparing available camera sensors on the market is very difficult. Most of the
available sensors are designed to be used in model making. The prices of the
ones that are intended for automotive applications (e.g. sensors of Ambarella
and ON Semiconductor) have to be inquired from the manufacturers. Table
\ref{tab:camera} shows two camera sensors that are suitable for a proof of
concept and that could also be tested under the condition of extreme
temperatures. 

\begin{table}
\centering
\captionsetup{justification=centering}
\footnotesize
\renewcommand{\arraystretch}{1.5}
\begin{tabular}{p{0.3\textwidth}|p{0.35\textwidth}|p{0.1\textwidth}|p{0.1\textwidth}}
Product & Key Features & Price & Retailer \\
\hline
32KM NTSC
& Resolution: 0.35MP \newline 
	Operation temp.: -20\degree C - 70\degree C \newline 
	Scanning freq.: 60Hz, \newline
	night vision: no, \newline
	angle: up to 120\degree
& 31.95\$ 
& \cite{sparkfun3} \\
LinkSprite JPEG Color Camera TTL Interface - Infrared
& Resolution: 0.3MP \newline 
	Operation temp.: -20\degree C - 70\degree C \newline 
	Scanning freq.: 60Hz, \newline
	night vision: yes, \newline
	angle: up to 120\degree
& 49.95\$ 
& \cite{sparkfun4}
\end{tabular}
\caption{Sampled Camera Sensors}
\label{tab:camera}
\end{table}

\newpage
\subsection{Calculation of Costs}
If a hardware proof of concept is implemented, the costs can be calculated like
it is done in table \ref{tab:costs}. The amount of sensors that are used in a
final system and the actual prices might differ in the moment of the
system's final realisation.

\begin{table}
\centering
\captionsetup{justification=centering}
\renewcommand{\arraystretch}{1.5}
\begin{tabular}{llll}
Part & Price/Unit & Pcs & Sum \\
\hline
32KM NTSC & 31.95\$ & 2 & \hspace{0.07cm} 63.90\$ \\
LV-MaxSonar-EZ1 & 23.36\$ & 4 & \hspace{0.07cm} 93.44\$ \\
 Arduino Mega 2560 & 37.18\$ & 1 & \hspace{0.07cm} 37.18\$ \\ 
\hline
Total & & & 194.52\$
\end{tabular}
\caption{Calculation of Costs for a POC}
\label{tab:costs}
\end{table}

\section{Conclusion}
The analysis of our company's competitors worked out that the planned product is
more sophisticated than those systems that are used in today's cars.
Nevertheless, the idea of autonomously leaving a parking space is not new and
there has been some research in this area at least since 2004.

The Robert Bosch GmbH is estimated to be the major competitor to our company
since they are well connected to important car manufacturers like Audi and
Mercedes-Benz. Bosch is involved in research projects that target on ``Valet
Parking'' and it also already demonstrated a product that is similar to the one
of our company using a real car. Due to the prognostication of the market's
development, there might however be the possibility to get a significant
market-share if the competitors like Bosch could be overcome by additional
functionality or improved safety and reliability.

The investigation of available hardware on the market gave a first estimation of
how much a hardware proof of concept might cost. It is certainly difficult to
derive the costs of the final product from this estimation since the used prices
pertain to the ordering of a small amount of sensors. Especially in the case of
camera sensors, there have to be some inquiries to the sensors' manufacturers to
get additional information about the capabilities and the prices of the offered
sensors.

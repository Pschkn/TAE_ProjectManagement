An important step in the development of a new product is to analyse the market.
This analysis not only includes the identification of competitors and their offered
technologies, but also the investigation of the demand on the product to develop
and its future progression.

The subsequent analysis is done in an indirect way so that the presented
information is retrieved by querying the internet and putting altogether the
relevant information.

\section{Current Systems implemented in today's Cars}
There are only a few systems available that help the driver in leaving a parking
space. These systems exhibit a huge variety of autonomy. The manufacturers
Volvo, Audi and Lincoln sheet park assistance systems that take control over the
steering wheel when leaving a parking lot (see \cite{VolvoCarsSupport},
\cite{Lincoln} and \cite{AudiEspana}). While the steering is done autonomously, 
the driver has to manually operate the pedals.
This kind of systems is mostly restricted to parallel parking.

Mercedes-Benz offers a more autonomous, but also more restricted way of assisted
parking. The �Mercedes Benz Parking Pilot� is able to park and leave the parking
site autonomously. The quitting of the parking site is restricted to those
scenarios in which the Parking Pilot was also used to park the car (see
\cite{DaimlerParkingPilot}).

Tesla offers the Summon functionality implemented in its Model S and Model X. It
allows a driver to leave its car and park as well as retrieve it autonomously.
This feature is restricted to perpendicular parking only (see \cite{TeslaSummon}).

\section{Current Systems available from Suppliers}

The development of systems assisting a driver in parking and leaving a parking
lot can be illustrated by the evolution of the products originating from Robert
Bosch GmbH. While the early systems act as it was described for the
manufacturers Volvo, Audi and Bosch (see \cite{Bosch2013}
),
 the current systems are now able to drive itself into and back out of a parking
site autonomously (see \cite{BoschFully}). Another future application of park
assistants is the �Bosch Home Zone Park Assist�. It enables a driver to train its car for certain parking situations
(see 
\cite{BoschHomeZone}).
The car records a route that is driven and it is able to reproduce it even if
the starting point of the route to drive and the one of the recorded route is
slightly different.
On its trained way, the car is able to detect impediments and to react to them.

\section{Scientific Projects}
There exist several projects that target on the functionality of autonomously
parking to and leaving from a parking lot.
While the work of Katsev and Braun (see
 \cite{Braun})
 that already started in 2004 seems not to have reached the point where leaving a parking lot is implemented since no further resources can be found on that project,
Roland Doloczki and Don Kevin Gaubitz produced a working prototype of RC-Car
that autonomously leaves a parking space (see
\cite{Doloczki}).
To achieve their goal, Doloczki and Gaubitz use ultrasonic and infrared sensors
to sense the environment around parked vehicle.

\section{Development of the market}
It is obvious that the demand on systems that perform certain manoeuvres
autonomously will increase with the success of autonomous cars. But also in the
meantime till these cars make the breakthrough, there might be an increased need
for \acf{ADAS} like parking assistants. Following
McKinsey Inc., there will be three eras in the revolution of self-driving cars
(see \cite{McKinsey}).
The first era, starting from now and lasting till the late 2020s, is
characterised by the first autonomous vehicles being produced and their impact
on established car manufacturers. McKinsey states that the premium makers will
take an incremental approach to autonomous vehicles by implementing more
sophisticated ADAS. This assumption is supported by Statista, assuming that the
shipment of ADAS units will increase by more than 500\% in the time from 2012 to
2020 (see \cite{StatisticaADAS}).

One of the buzz word regarding future driver assistance systems is �Valet
Parking� which means that a car parks itself after the driver has left it and
that the car can be retrieved from its parking position without active control
of the driver. Therefore, Valet Parking needs the possibility of a car
autonomously leaving its parking site. A research project targeting on Valet
Parking was announced by Daimler, Bosch and Car2go in the year 2015 (see
\cite {DaimlerValet}).

\section{Conclusion}
It has been worked out that the systems that are implemented in today�s cars are
less sophisticated than the system that is planned to develop. Additionally, the
increasing need for ADAS like park assistants has been exposed. However, there
are other scientific projects that aim on the same kind of system and that have
to be overcome by additional functionality or improved safety and reliability.
The major competitor in this sector will be the Robert Bosch GmbH that already
demonstrated its product with a real vehicle and that is working together with a
lot of important car manufacturers like Daimler or Audi.

\section{Conclusion - Markus Just}

\section{Conclusion - Timo Acquistapace}
This project was the first one in my educational experience that included a
detailed process of project management. As a consequence, it was on the one hand
very interesting to get an insight into the steps that could help to run a
project successfully. On the other hand, there are some points that would be
done in a different way the next time a similar project is set up, e.g. the
assignment of roles. The roles were initially assigned to the team members with
regard to their technical skills and knowledge.  Since every team member was
also assigned to do some work in the area of project management, there occurred
some interdependencies that were not considered in the assignment of roles. An
exemplary interdependence in the current project is the following: To evaluate
the customer's satisfaction with the first drafts of the system, it is very
helpful if a design sketch is available. However, the designer was not entrusted
with the task of evaluating the customer's satisfaction. To avoid additional
overhead for synchronization and waiting times, the initial assignment of roles
was defined down and the creation of mockups as well as their enhancement
according to the customer's feedback was assigned to the person that was
responsible for the involvement of the customer.

Embedding the customer's voice itself is proved itself expensive. After having
collected some experience with agile methods and their regular, but non-formal
events to collect feedback on a product or process, the way of using
questionnaires to determine the customer's satisfaction appears nonelastic and
complicated at the first glance. But over the course of the project, the
advantages of documented feedback became clear. The used tool to create the
questionnaires -- Satistica -- proved itself very helpful. Gathered feedback is
diagrammed clearly by Satistica and single questions as well as whole sections
of questions can be evaluated easily regarding the point if the answers fit the
expectations. It is therefore very easy to determine the areas of work where
some improvements have to take place. Nevertheless, written feedback that does
not offer the possibility to make a query directly could also be misunderstood
like every written document. It has to be kept in mind that creating good
questionnaires is a part of social science and there are many points to
consider. Concluding, I would value direct discussions over feedback
questionnaires. If this is not possible, Satistica anyway offers good support in
collecting a customer's feedback.

The market analysis conducted exhibits the greatest deviation from the way it
would be done in a commercial project. Whilst the prices of needed hardware
parts are only gathered by a research on the internet, contacting the different
manufacturer would be inevitable in a commercial project. The manufacturers may
provide more favourable conditions if a larger amount of pieces is ordered and
they may also counsel our company on the selection of the products. If the
analysis is done in a very early stage, even before the project team is chosen,
the persons that already published research projects like Katsev and Braun might
be contacted and asked to contribute to the project by bringing in their
experience. The greatest weakness of the conducted market analysis is the way
how the competitors' products were analysed. While the information was again
gathered by research on the internet, a commercial project would require to test
the products that are already available on the market. Only this it is possible
to get an impression of how these product might be improved and how the
competitors could thereby be overcome.

\section{Conclusion - Simon Schneider}

In my own opinion it was a quite interesting project, despite the fact that it
was the first time I worked on a project which includes such an effort on
project management. However, the project helped a lot to collect more experience
and to better understand the manner of project management. Since no one of the
project members did such a project before, the start of the project was a
little bit difficult. The main problem at the beginning was the distribution of
roles, since no one knew exactly what to do in the respective roles and the
related tasks. Hence, with the new gained experience, this could be improved in
the future and applied on another project. The communication between the project
members was quite satisfying and the working process was getting better after
the bumpy start. Also because each project participant got his role assigned and
now knew what the tasks should contain.

The realisation of a risk analysis for such a project was also new to this 
extent. Although it was quite hard to figure out, what tasks are included in a
risk analysis and what is important for this kind of project, a risk analysis is
very helpful in terms of facing the threats which might occure in a project.
For that purpose, a risk analysis should be done as early as possible to 
anticipate and neutralise possible problems when planning the project.Another
outcome of that analysis is to decide wether to continue with the project or not
and the most important to improve the safety and to deal with potential risks of
the whole project. The chapter exhibited how to identify risk that can occur and
how to estimate the risk value. Afterwards it was demonstrated and explained how
risks can be prioritised in terms of their impact on the project. At the end
four different ways of dealing with risks has been dealt with.

Summarised, it was a challenging but a very good experience to work on this
project, although there has been lots of problems to be solved. Certainly there
is some space for furhter improvements but especially with such an unexperienced
team it was nice to see the furhter the project progressed the better the team
worked together.

\section{Conclusion - Wojciech Lesnianski} 
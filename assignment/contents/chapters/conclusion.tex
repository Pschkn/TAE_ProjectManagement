\section{Conclusion - Markus Just}
Working on this project has provided some useful insights into the management
aspects of a professional project. The major difficulty, as the project manager,
was to organize the distribution of the tasks, espescially the reporting and the
project management tasks, because each team member wanted to take part on the
management aspect, which is the central aspect in this module.  Another
difficulty was to distribute the tasks over the large amount of time. We decided
that we do the planning intially in December and the real project work should
start in March, so that it is easier to communicate if everyone is working on
the project at the same time, but this wasn't the optimal approach. It would
have been more effective if the project work could be done full-time within a
smaller time period, but in our case this wasn't possible due to company work
and other master courses/assignments.

My major task was the project organising and the requirements/specification.
While the second aspect was relatively straight-forward, the organising was a
lot of work. Creating the task and resource plan took a large amount of time and
things had to be changed very often, because some tasks became obsolet and some
tasks had to be added during the project work phase.  Another difficult planning
aspect was to find the dependencies between tasks to generate a meanignful task
overview with the possibility to identify the critical path. All in all, the
effort for planning this relatively small project was surprisingly high and my
respect for the people who are responsible for this in really large professional
projects did definitely grow.

The communication within the team and the communication with the customer went
really well, which made the project work comfortable. Of course, the aspect that
we have known each other before helped here, but nevertheless I am thankful that
it worked that well.

All in all, I learned some important things about project management, which will
definitely help in my future career, but I'm also happy that I am ``just'' a
developer at the moment and that I am the one who have to care more about
technical stuff than about planning and organizing. But with more experience
this may change in a couple of years and then the things I've learned here could
be helpful.


\newpage
\section{Conclusion - Timo Acquistapace}
This project was the first one in my educational experience that included a
detailed process of project management. As a consequence, it was on the one hand
very interesting to get an insight into the steps that could help to run a
project successfully. On the other hand, there are some points I would do in a
different way the next time a similar project is set up, e.g. the assignment 
of roles. The roles were initially assigned to the team members with
regard to their technical skills and knowledge.  Since every team member was
also assigned to do some work in the area of project management, there occurred
some interdependencies that were not considered in the assignment of roles. An
exemplary interdependence in the current project is the following: To evaluate
the customer's satisfaction with the first drafts of the system, it is very
helpful if a design sketch is available. However, the designer was not entrusted
with the task of evaluating the customer's satisfaction. To avoid additional
overhead for synchronization and waiting times, the initial assignment of roles
was defined down and the creation of mockups as well as their enhancement
according to the customer's feedback was assigned to the person that was
responsible for the involvement of the customer.

Embedding the customer's voice proved itself expensive. After having
collected some experience with agile methods and their regular, but non-formal
events to collect feedback on a product or process, the way of using
questionnaires to determine the customer's satisfaction appears nonelastic and
complicated at the first glance. But over the course of the project, the
advantages of documented feedback became clear. The used tool to create the
questionnaires -- Satistica -- proved itself very helpful. Gathered feedback is
diagrammed clearly by Satistica and single questions as well as whole sections
of questions can be evaluated easily regarding the point if the answers fit the
expectations. It is therefore very easy to determine the areas of work where
some improvements have to take place. Nevertheless, written feedback that does
not offer the possibility to make a query directly could also be misunderstood
like every written document. It has also to be kept in mind that creating good
questionnaires is part of social sciences and there are many points to
consider. Concluding, I would value direct discussions over feedback
questionnaires. If this is not possible, Satistica anyway offers good support in
collecting a customer's feedback.

The market analysis conducted exhibits the greatest deviation from the way it
would be done in a commercial project. Whilst the prices of needed hardware
parts are only gathered by a research on the internet, contacting the different
manufacturers would be inevitable in a commercial project. The manufacturers may
provide more favourable conditions if a larger amount of pieces is ordered and
they may also counsel our company on the selection of the products. If the
analysis is done in a very early stage, even before the project team is chosen,
the persons that already published research projects like Katsev and Braun might
be contacted and asked to contribute to the project by bringing in their
experience. The greatest weakness of the conducted market analysis is the way
how the competitors' products were analysed. While the information was again
gathered by research on the internet, a commercial project would require to test
the products that are already available on the market. Only this way it is
possible to get an impression of how these products might be improved and how
the competitors could thereby be overcome.

\newpage
\section{Conclusion - Simon Schneider}

In my own opinion it was a quite interesting project, despite the fact that it
was the first time I worked on a project which includes such an effort on
project management. However, the project helped a lot to collect more experience
and to better understand the manner of project management. Since no one of the
project members did such a project before, the start of the project was a
little bit difficult. The main problem at the beginning was the distribution of
roles, since no one knew exactly what to do in the respective roles and the
related tasks. Hence, with the new gained experience, this could be improved in
the future and applied on another project. The communication between the project
members was quite satisfying and the working process was getting better after
the bumpy start. Also because each project participant got his role assigned and
now knew what the tasks should contain.

The realisation of a risk analysis for such a project was also new to this 
extent. Although it was quite hard to figure out, what tasks are included in a
risk analysis and what is important for this kind of project, a risk analysis is
very helpful in terms of facing the threats which might occure in a project.
For that purpose, a risk analysis should be done as early as possible to 
anticipate and neutralise possible problems when planning the project.Another
outcome of that analysis is to decide wether to continue with the project or not
and the most important to improve the safety and to deal with potential risks of
the whole project. The chapter exhibited how to identify risk that can occur and
how to estimate the risk value. Afterwards it was demonstrated and explained how
risks can be prioritised in terms of their impact on the project. At the end
four different ways of dealing with risks has been dealt with.

Summarised, it was a challenging but a very good experience to work on this
project, although there has been lots of problems to be solved. Certainly there
is some space for furhter improvements but especially with such an unexperienced
team it was nice to see the furhter the project progressed the better the team
worked together.

\newpage
\section{Conclusion - Wojciech Lesnianski} 
During the description of our system -- a system that should support the user by
taking out a car out of a parking position -- and the planning of the
development process, it was interesting to see, at how many places experience
would be the most helpful tool. Even the smallest decision in the beginning can
have a big impact on the planning and the development process, so every decision
should be thought through. Assigning an unfitting task to the role of a team
member can have a big impact not only on his own productivity, but also on the
productivity of the whole team, if it means he is blocking someone else. The
process of finding the right person to do a job is only possible, if one keeps a
lot of things in the back of his head. It is not easy to decide whether a person
has the needed expertise if you never did a project with that person before. You
need to know exactly what skills will be needed and to what degree to evaluate a
candidate. The less resources are available, the better the planning must be --
the higher of an impact a wrong decision can make.

Another important question is, how well the customer knows his own needs and how
well he can specify them. If requirements are clear and understandable, writing
a specification can be a pleasant task. If requirements are full of gaps and
overall incomplete, a gap analysis and communication with the customer are
crucial. Errors done in the planning phase are the most critical ones, since
they are also the most expensive ones. This is the reason why agile methods
became so popular � the more the customer gets included in the planning and
development process, the smaller the chances are, that a misunderstanding slips
through into the final product, or gets recognized at a much later phase.

Overall this assignment also showed me, that my strengths and interests don't
lie in complex art of project planning or managing, but in the developer role.
Maybe after getting some experience with Projects from the developer side it
will become a different story, but for now I would much rather prefer to follow
a master plan of someone experienced.
